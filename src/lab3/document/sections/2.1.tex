\graphicspath{{images/act_2.1/}}
\subsection{Cartesian space inverse dynamics}
The objective of this activity is to control movement of the ur5 robot end-effector so that it follows the Cartesian sinusoidal reference trajectory of activity \ref{subsec:generate_sinusoidal_reference}. The simulation starts with initial joint configuration $\mathbf{q_0}=\begin{bmatrix} 0.0 & -1.0 & 1.0 & 0.5 & 0.0 & 0.5 \end{bmatrix}$ rad and end-effector $\mathbf{p_0}=\begin{bmatrix}  0.577 &   0.192 &   0.364 \end{bmatrix}$~m. Likewise, the Cartesian sinusoidal reference trajectory starts at $\mathbf{p_0}$. Motion control is made up of two approaches: Cartesian proportional-derivative with gravity compensation, feed-forward term (PD+g+ff) and projection of the null space. In this sense, Cartesian PD+g+ff focuses on reducing end-effector position error and the projection of null space maintains the articular position close to $\mathbf{q_0}$. Finally, control law can be computed as 
\begin{equation}
	\boldsymbol{\tau}
	= \mathbf{J^T} (\boldsymbol{\Lambda}\mathbf{\ddot{p}_{des}} +\mathbf{K_p e} + \mathbf{K_d \dot{e}}+ \mathbf{J^{T\#} g}) + \mathbf{N} \left(\mathbf{K_q(q_0-q) - D_q \dot{q}} \right),
	\label{eq:cartesian_PD_g_ff_N}
\end{equation}
\begin{equation*}
	\mathbf{N}=(\mathbf{I_{6 \times 6}} - \mathbf{J^{\#} J} ),
\end{equation*}
\begin{equation*}
	\boldsymbol{\Lambda}= (\mathbf{J M^{-1} J^{T}})^{-1},
\end{equation*}
\noindent where $\mathbf{J}$ is jacobian matrix, $\boldsymbol{\Lambda}$ is inertia matrix at Cartesian space, $\mathbf{e}=\mathbf{p_{des} - p}$ is end-effector position error, $\mathbf{K_p, K_d}$ are the proportional and derivative gains respectively, $\mathbf{J^{\#}}$ is jacobian damped pseudo-inverse, $\mathbf{g}$ is gravity compensation term, $\mathbf{N}$ is the null space projection of $\mathbf{J^{\#}}$, and $\mathbf{K_q, D_q}$ are the proportional and derivative gains for null space projection. \vspace{.5cm}